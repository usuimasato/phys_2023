\documentclass[uplatex,dvipdfmx, a4paper,11pt]{jsarticle}
% 数式
\usepackage{amsmath,amsfonts,amssymb}
\usepackage{mathrsfs}
\usepackage{mathtools}
\usepackage{latexsym}
\usepackage{bm}
% 画像
\usepackage[dvipdfmx]{graphicx}
\usepackage{enumerate}
\usepackage{setspace}
\usepackage{physics}
\usepackage{multicol}
\usepackage{ascmac}
\usepackage[dvipdfmx]{hyperref}
\usepackage{pxjahyper}
\usepackage{fancybox}
\usepackage[top=1cm, bottom=2cm, right=2.5cm, left=2.5cm]{geometry}
\usepackage{tikz}
\usepackage{color}
\usepackage{tcolorbox}
\usepackage{framed}
\usepackage{ulem}
\hypersetup{%
colorlinks=true,%
linkcolor=blue,%
citecolor=red}
\usetikzlibrary{intersections,calc,arrows.meta}
\begin{document}

\title{量子力学II第1回レポート問題}
\author{Masato USUI}
\date{\today}
\maketitle
\begin{enumerate}[1.]
    \item \begin{enumerate}[1)]
        \item $d=2$の時\\
    \begin{equation}
        \ket{j=1/2,m=1/2}\doteq \begin{pmatrix}
            1\\0
        \end{pmatrix}
        ,\quad \ket{j=1, m=-1/2}\doteq \begin{pmatrix}
            0\\1
        \end{pmatrix}
    \end{equation}
    とする.
        明らかに
    \begin{equation}
        M_z\doteq \frac{\hbar}{2}\begin{pmatrix}
            1&0\\
            0&-1
        \end{pmatrix}
    \end{equation}
    である.また$M_{\pm}=M_x\pm iM_y$と置くと角運動量演算子の定義から
    \begin{align*}
        \qty[M_z, M_{\pm}]&=\qty[M_z,M_x\pm iM_y]\\
        &=i\hbar\qty(M_y\mp iM_x)\\
        &=\pm\hbar\qty(M_x\pm iM_y)\\
        &=\pm\hbar M_{\pm}
    \end{align*}
    であるから
    \begin{align*}
        M_zM_{\pm}\ket{j=1/2,m=\pm 1/2}&=M_{\pm}M_z\ket{j=1/2,m=\pm 1/2}\pm \hbar M_{\pm}\ket{j=1/2,m=\pm 1/2}\\
        &=M_{\pm}\qty(m\pm \hbar)\ket{j=1/2, m=\pm 1/2}
    \end{align*}
    つまり$M_{\pm}\ket{j,m}$は$M_z$の固有状態または0ベクトルである
    \begin{align*}
        M_{+}\ket{j,m}\propto  &\begin{cases}
            \ket{0}&\qq{$m$=1/2}\\
            \ket{j=1/2,m=1/2} &\qq{$m=-1/2$}
        \end{cases}\\
        M_{-}\ket{j,m}\propto &\begin{cases}
            \ket{j=1/2,m=-1/2}&\qq{$m=1/2$}\\
            \ket{0}&\qq{$m=-1/2$}
        \end{cases}
    \end{align*}
    さらにこの比例係数は
    $M_{\pm}^{\dagger}M_{\pm}=M^2-M_z^2\mp\hbar M_z$であることを用いると
    \begin{align*}
        \abs{c_{+}}^2&=\ev{M_{+}^{\dagger}M_{+}}{j=1/2, m=-1/2}\\
        &=\qty(\frac{\hbar}{2}\frac{3\hbar}{2}-\frac{\hbar^2}{4}+\frac{1}{2}\hbar^2)\\
        &=\hbar^2\\
        \abs{c_{-}}^2&=\ev{M_{-}^{\dagger}M_{-}}{j=1/2, m=1/2}\\
        &=\qty(\frac{\hbar}{2}\frac{3\hbar}{2}-\frac{\hbar^2}{4}+\frac{\hbar^2}{2})\\
        &=\hbar^2
    \end{align*}
    であるから$c_{+}=c_{-}=\hbar$にとる.すると結局
    \begin{equation}
        M_{+}\doteq \hbar\begin{pmatrix}
            0&1\\
            0&0
        \end{pmatrix}
        ,\quad M_{-}\doteq \hbar\begin{pmatrix}
            0&0\\
            1&0
        \end{pmatrix}
    \end{equation}
    $M_x=1/2(M_{+}+M_{-}),\ M_y=1/2i(M_{+}-M_{-})$であることから
    \begin{equation}
        M_x\doteq \frac{\hbar}{2}\begin{pmatrix}
            0&1\\
            1&0
        \end{pmatrix}
        ,\quad M_y\doteq \frac{\hbar}{2}\begin{pmatrix}
            0&-i\\
            i&0
        \end{pmatrix}
    \end{equation}
    \begin{equation}
        (M_x,M_y,M_z)\doteq\frac{\hbar}{2}\qty(\begin{pmatrix}
            0&1\\1&0
        \end{pmatrix},\begin{pmatrix}
            0&-i\\i&0
        \end{pmatrix},\begin{pmatrix}
            1&0\\0&-1
        \end{pmatrix})
    \end{equation}
    これは定数倍を除いてパウリ行列と一致する.\vspace{5mm}\\
     $d=3$の時
    \begin{equation}
        \ket{j=1,m=1}\doteq \begin{pmatrix}
            1\\0\\0
        \end{pmatrix},\quad
        \ket{j=1,m=0}\doteq \begin{pmatrix}
            0\\1\\0
        \end{pmatrix},\quad
        \ket{j=1,m=-1}\doteq \begin{pmatrix}
            0\\0\\1
        \end{pmatrix}
    \end{equation}
    とする.こちらも明らかに
    \begin{equation}
        M_z\doteq \hbar\begin{pmatrix}
            1&0&0\\
            0&0&0\\
            0&0&-1
        \end{pmatrix}
    \end{equation}
    である.先ほどと同様に$M_{\pm}$を定義すると上と同様の議論によって
    \begin{align*}
        M_{+}\ket{j,m}&\propto \begin{cases}
            \ket{0}&\qq{$m=1$}\\
            \ket{j=1,m=1}&\qq{$m=0$}\\
            \ket{j=1, m=0}&\qq{$m=-1$}
        \end{cases}\\
        M_{-}\ket{j,m}&\propto \begin{cases}
            \ket{j=1,m=0}&\qq{$m=1$}\\
            \ket{j=1,m=-1}&\qq{$m=0$}\\
            \ket{0}&\qq{$m=-1$}
        \end{cases}
    \end{align*}
    である.各係数は$M_{\pm}^{\dagger}M_{\pm}$の演算子をかけることで
    \begin{align*}
        \abs{c_{+m=0}}^2&=\ev{M_{+}^{\dagger}M_{+}}{j=1,m=0}\\
        &=\qty(\hbar \cdot2\hbar)\\
        &=2\hbar^2\\
        \abs{c_{+m=-1}}^2&=\ev{M_{+}^{\dagger}M_{+}}{j=1,m=-1}\\
        &=\qty(\hbar\cdot2\hbar-\hbar^2+\hbar^2)\\
        &=2\hbar^2\\
        \abs{c_{-m=1}}^2&=\ev{M_{-}^{\dagger}M_{-}}{j=1,m=1}\\
        &=\qty(2\hbar^2-\hbar^2+\hbar^2)\\
        &=2\hbar^2\\
        \abs{c_{m=0}}^2&=\ev{M_{-}^{\dagger}M_{-}}{j=1,m=0}\\
        &=2\hbar^2
    \end{align*}
    であるから全て$c=\sqrt{2}\hbar$にとる.すると$M_{+},M_{-}$の行列表現は
    \begin{equation}
        M_{+}\doteq \sqrt{2}\hbar\begin{pmatrix}
            0&1&0\\
            0&0&1\\
            0&0&0
        \end{pmatrix},\quad 
        M_{-}\doteq \sqrt{2}\hbar\begin{pmatrix}
            0&0&0\\
            1&0&0\\
            0&1&0
        \end{pmatrix}
    \end{equation}
    である.これから$M_x, M_y$を求めると
    \begin{equation}
        M_x=\frac{\hbar}{\sqrt{2}}\begin{pmatrix}
            0&1&0\\
            1&0&1\\
            0&1&0
        \end{pmatrix},\quad 
        M_y=\frac{\hbar}{\sqrt{2}}\begin{pmatrix}
            0&-i&0\\
            i&0&-i\\
            0&i&0
        \end{pmatrix}
    \end{equation}
    となる.
    \begin{equation}
        (M_x,M_y,M_z)\doteq\frac{\hbar}{\sqrt{2}}\qty(\begin{pmatrix}
            0&1&0\\1&0&1\\0&1&0
        \end{pmatrix}, \begin{pmatrix}
            0&-i&0\\i&0&-i\\0&i&0
        \end{pmatrix},\begin{pmatrix}
            \sqrt{2}&0&0\\0&0&0\\0&0&-\sqrt{2}
        \end{pmatrix})
    \end{equation}
        \item 角運動量の合成によって得られる$j$は$j_{max}=j_A+j_B=1$, $j_{min}=\abs{h_A-j_B}=0$であるから合成後の角運動量はこの2つ.
        \begin{enumerate}[(a)]
            \item $j=1$の時
                とりうる$m$は$m=1,0,-1$のいずれか.まずは$\ket{j_A=1/2,j_B=1/2,j=1,m=1}$について. $m=m_A+m_B$より
                \begin{equation}
                    \ket{j_A=1/2,j_B=1/2;j=1,m=1}=\ket{j_A=1/2,m_A=1/2}\otimes \ket{j_B=1/2, m_B=1/2}
                \end{equation}
                である. $J^{-}$をかけることで$m=0$の場合は
                \begin{align*}
                    \sqrt{2}\hbar\ket{j_A=1/2,j_B=1/2,j=1,m=0}=&\sqrt{\frac{3}{4}+\frac{1}{4}}\hbar\ket{j_A=1/2,m_B=-1/2}\otimes\ket{j_B=1/2,m_B=1/2}\\
                    &+\sqrt{\frac{3}{4}+\frac{1}{4}}\hbar\ket{j_A=1/2,m_A=1/2}\otimes\ket{j_B=1/2, m=-1/2} 
                \end{align*}
                であるからこれを整理することで
                \begin{equation}
                    \ket{1/2,1/2;1,0}=\frac{1}{\sqrt{2}}\qty(\ket{1/2,-1/2}\otimes \ket{1/2,1/2}+\ket{1/2,1/2}\otimes\ket{1/2,-1/2})
                \end{equation}
                $m=-1$に場合も$J^{-}$をかけると
                \begin{align*}
                    \sqrt{2}\hbar\ket{1/2,1/2,1,-1}=&\frac{1}{\sqrt{2}}\sqrt{\frac{3}{4}+\frac{1}{4}}\hbar\ket{1/2,-1/2}\otimes\ket{1/2,-1/2}\\&+\frac{1}{\sqrt{2}}\sqrt{\frac{3}{4}+\frac{1}{4}}\ket{1/2,-1/2}\otimes\ket{1/2,-1/2}
                \end{align*}
                これを整理して
                \begin{equation}
                    \ket{1/2,1/2,1,-1}=\ket{1/2,-1/2}\otimes\ket{1/2,-1/2}
                \end{equation}
            \item $j=0$の時とりうる$m$は$m=0$のみである.よって$\ket{1/2,1/2;0,0}$は$\alpha\ket{1/2,1/2}\otimes\ket{1/2,-1/2}$と$\beta\ket{1/2,-1/2}\otimes\ket{1/2,1/2}$の和でかける. $j=1$の状態と直交しなければならないから特に$\ket{1/2,-1/2}\otimes\ket{1/2,1/2}+\ket{1/2,1/2}\otimes\ket{1/2,-1/2}$との内積を考えることで$\alpha+\beta=0$,これと正規化条件$\abs{\alpha}^2+\abs{\beta}^2=1$を考慮すると全体の位相を除いて$\alpha=1/\sqrt{2}, \beta=-1/\sqrt{2}$と定まる.すなわち
            \begin{equation}
                \ket{1/2,1/2;1,0}=\frac{1}{\sqrt{2}}\qty(\ket{1/2,1/2}\otimes\ket{1/2,-1/2}-\ket{1/2,-1/2}\otimes\ket{1/2,1/2})
            \end{equation}
            である.
        \end{enumerate}
        \item $s^A\otimes I+I\otimes s^B$は
        \begin{align*}
            s^A\otimes I+I\otimes s^B&\doteq \qty(\frac{\hbar}{2}\begin{pmatrix}
                0I_B&1I_B\\
                1I_B&0I_B
            \end{pmatrix}+\frac{\hbar}{2}
            \begin{pmatrix}
                    0I_B&-iI_B\\
                    iI_B&0
            \end{pmatrix}
                +\frac{\hbar}{2}\begin{pmatrix}
                1I_B&0I_B\\
                0I_B&-1I_B
            \end{pmatrix}+\begin{pmatrix}
                1s^B&0s^B\\
                0s^B&1s^B
            \end{pmatrix})
        \end{align*}
        これを2乗すると
        \begin{align*}
            (s^A\otimes I+I\otimes s^B)^2\doteq\frac{\hbar^2}{4}\qty{\begin{pmatrix}
                0&1&1&0\\
                1&0&0&1\\
                1&0&0&1\\
                0&1&1&0
            \end{pmatrix}^2+\begin{pmatrix}
                0&-i&-i&0\\
                i&0&0&-i\\
                i&0&0&-i\\
                0&i&i&0   
            \end{pmatrix}^2+
            \begin{pmatrix}
                2&0&0&0\\
                0&0&0&0\\
                0&0&0&0\\
                0&0&0&-2
            \end{pmatrix}^2
            }\equiv X
        \end{align*}
        これを計算すると
        \begin{equation}
            X=\hbar^2\begin{pmatrix}
                2&0&0&0\\
                0&1&1&0\\
                0&1&1&0\\
                0&0&0&2
            \end{pmatrix}
        \end{equation}
        ただし基底は
        \begin{gather*}
            \ket{1/2,1/2}\otimes\ket{1/2,1/2}\doteq\begin{pmatrix}
                1\\0\\0\\0
            \end{pmatrix}\quad 
            \ket{1/2,1/2}\otimes \ket{1/2,-1/2}\doteq\begin{pmatrix}
                0\\1\\0\\0
            \end{pmatrix}\\
            \ket{1/2,-1/2}\otimes\ket{1/2,1/2}\doteq \begin{pmatrix}
                0\\0\\1\\0
            \end{pmatrix}\quad
            \ket{1/2,-1/2}\otimes \ket{1/2,-1/2}\doteq \begin{pmatrix}
                0\\0\\0\\1
            \end{pmatrix}            
        \end{gather*}
        である.ここで基底の変換行列$P$を
        \begin{equation}
            P=\begin{pmatrix}
                1&0&0&0\\
                0&\frac{1}{\sqrt{2}}&\frac{1}{\sqrt{2}}&0\\
                0&\frac{1}{\sqrt{2}}&-\frac{1}{\sqrt{2}}&0\\
                0&0&0&1
            \end{pmatrix}
        \end{equation}
        で定める.これ2)における$j$に対する固有ベクトルを与える基底に変換する行列である.また$P^{-1}=P$であることは容易に確かめられる.
        \begin{align*}
            PXP^{-1}&=\hbar^2\begin{pmatrix}
                1&0&0&0\\
                0&\frac{1}{\sqrt{2}}&\frac{1}{\sqrt{2}}&0\\
                0&\frac{1}{\sqrt{2}}&-\frac{1}{\sqrt{2}}&0\\
                0&0&0&1
            \end{pmatrix}\begin{pmatrix}
                2&0&0&0\\
                0&2&1&0\\
                0&1&2&0\\
                0&0&0&2
            \end{pmatrix}\begin{pmatrix}
                1&0&0&0\\
                0&\frac{1}{\sqrt{2}}&\frac{1}{\sqrt{2}}&0\\
                0&\frac{1}{\sqrt{2}}&-\frac{1}{\sqrt{2}}&0\\
                0&0&0&1
            \end{pmatrix}\\
            &=\hbar^2\begin{pmatrix}
                2&0&0&0\\
                0&2&0&0\\
                0&0&0&0\\
                0&0&0&2
            \end{pmatrix}
        \end{align*}
        これは2)で求めた基底での行列表現であり,2)のベクトルが固有ベクトルになっていることが分かる.
    \end{enumerate}
    \item \begin{enumerate}[1)]
        \item 軌道角運動量は3階完全反対称テンソル$\varepsilon_{ijk}$を用いて
        \begin{equation}
            L_i=-i\hbar\sum_{j,k}\varepsilon_{ijk}x^{j}\pdv{x^k}
        \end{equation}
        でかける. $L_i, L_j$の交換関係は
        \begin{align*}
            [L_i,L_j]&=(L_iL_j-L_jL_i)\\
            &=-\hbar^2\sum_{klmn}\varepsilon_{ikl}\varepsilon_{jmn}\qty(x^k\pdv{x^l}\qty(x^m\pdv{x^n})-x^m\pdv{x^n}\qty(x^k\pdv{x^l}))\\
            &=-\hbar^2\sum_{klmn}\varepsilon_{ikl}\varepsilon_{jmn}\qty(x^k\delta_{ml}\pdv{x^n}-x^m\delta_{kn}\pdv{x^l})\\
            &=-\hbar^2\qty{\sum_{kln}\varepsilon_{ikl}\varepsilon_{jln}x^k\pdv{x^n}-\sum_{klm}\varepsilon_{ikl}\varepsilon_{jmk}x^m\pdv{x^l}}\\
            &=-\hbar^2\qty{-\sum_{kn}(\delta_{ij}\delta_{kn}-\delta_{in}\delta_{kj})x^k\pdv{x^n}+\sum_{lm}(\delta_{ij}\delta_{lm}-\delta_{im}\delta_{jl})x^m\pdv{x^l}}\\
            &=-\hbar^2\qty{\delta_{ij}\sum_{l}x^l\pdv{x^l}-x^i\pdv{x^j}-\delta_{ij}\sum_{k}x^k\pdv{x^k}+x^j\pdv{x^i}}\\
            &=i\hbar\cdot\qty{i\hbar\qty(x^j\pdv{x^i}-x^i\pdv{x^j})}\\
            &=i\hbar\qty(i\hbar\sum_{mn}(\delta_{in}\delta_{jm}-\delta_{jn}\delta_{im})x^m\pdv{x^n})\\
            &=i\hbar\qty(i\hbar\sum_{k,m,n}\varepsilon_{kij}\varepsilon_{knm}x^m\pdv{x^n})\\
            &=i\hbar\sum_{k}\varepsilon_{kij}L_k
        \end{align*}
        \item 
        $v_i,\ v_j$を$(x,y,z)$で表示する:
        \begin{equation*}
            \bm{v}^x=
            \begin{pmatrix}
                n_{x1}\\n_{y1}\\n_{z1}
            \end{pmatrix}
            \bm{v}^y=
            \begin{pmatrix}
                n_{x2}\\n_{y2}\\n_{z2}
            \end{pmatrix}
            \bm{v}^z=
            \begin{pmatrix}
                n_{x3}\\n_{y3}\\n_{z3}
            \end{pmatrix}=
            \begin{pmatrix}
                n_{y1}n_{z2}-n_{z1}n_{y2}\\n_{z1}n_{x2}-n_{x1}n_{z2}\\n_{x1}n_{y2}-n_{y1}n_{x2}
            \end{pmatrix}
        \end{equation*}
        である.定義から明らかに
        \begin{equation}
            \bm{v}^i\times \bm{v}^j=\varepsilon_{ijk}\bm{v}^k
        \end{equation}
        \begin{align*}
            \qty[L_i^{\prime},L_j^{\prime}]&=\qty[n_{xi}L_x+n_{yi}L_y+n_{zi}L_z,n_{xj}L_x+n_{yj}L_y+n_{zj}L_z]\\
            &=i\hbar\qty{(n_{xi}n_{yj}-n_{yi}n_{xj})L_z+(n_{zi}n_{xj}-n_{xi}n_{zj})L_y+(n_{yi}n_{zj}-n_{zi}n_{yj})L_x}\\
            &=i\hbar(\bm{v}^i\times\bm{v}^j)\cdot\bm{L}\\
            &=i\hbar\varepsilon_{ijk}\bm{v}\cdot\bm{L}\\
            &=i\hbar\varepsilon_{ijk} L_k^{\prime}
        \end{align*}
        \item 実際に計算すると
        \begin{align*}
            (L_x^{\prime})^2+(L_y^{\prime})^2+(L_z^{\prime})^2&=(n_{x1}^2+n_{x2}^2+n_{x3}^2)L_x^2+(n_{y1}^2+n_{y2}^2+n_{y2}^2)L_y^2+(n_{z1}^2+n_{z2}^2+n_{z3}^2)L_z^2
        \end{align*}
        ここで
        \begin{align*}
            n_{x1}^2+n_{x2}^2+n_{x3}^2&=n_{x1}^2+n_{x2}^2+(n_{y1}n_{z2}-n_{z1}n_{y2})^2\\
            &=n_{x1}^2+n_{x2}^2+n_{y1}^2n_{z2}^2+n_{z1}^2n_{y2}^2-2n_{y1}n_{y2}n_{z1}n_{z2}
        \end{align*}
        さらに$\bm{v}^x,\bm{v}^y$の直交性から
        \begin{gather*}
            n_{x1}n_{x2}+n_{y1}n_{y2}+n_{z1}n_{z2}=0\\
            -2n_{y1}n_{y2}n_{z1}n_{z2}=-n_{x1}^2n_{x2}^2+n_{y1}^2n_{y2}^2+n_{z1}^2n_{z2}^2
        \end{gather*}
        であるから
        \begin{align*}
            n_{x1}^2+n_{x2}^2+n_{x3}^2&=n_{x1}^2+n_{x2}^2-n_{x1}^2n_{x2}^2+n_{y1}^2(n_{z2}^2+n_{y2}^2)+n_{z1}^2(n_{y2}^2+n_{z2}^2)\\
            &=n_{x1}^2+n_{x2}^2(n_{y1}^2+n_{z1}^2)+n_{y1}^2(1-n_{x2}^2)+n_{z1}^2(1-n_{x2}^2)\\
            &=n_{x1}^2+n_{y1}^1+n_{z1}^2=1
        \end{align*}    
        他の成分も同様だから
        \begin{equation}
            (L_x^{\prime})^2+(L_y^{\prime})^2+(l_z^{\prime})^2=L_x^2+L_y^2+L_z^2=L^2
        \end{equation}
    \end{enumerate}

\end{enumerate}




\end{document}